\documentclass{article}
\usepackage{projstyle}
\usepackage{graphicx}
\usepackage{mathtools}
\usepackage{enumitem}

\title{A Brief Introduction to Legendrian Knots}
\author{Shreyas Casturi\\Ana Fishburn}
\date{5 November 2018}

\begin{document}

\maketitle

\begin{abstract}
We start by defining Legendrian knots and isotopies of Legendrian knots.
Then we study the classical invariants of Legendrian knots:
the topological knot type, the Thurston-Bennequin number, and the rotation number.
We then explain how, while sufficient for certain classes of knots in tight
contact structures, there are nonisotopic Legendrian knots with equal
classical invariants.

In particular, we introduce the Chekanov-Eliashberg DGA, and see that its
homology, the knot contact homology, is an invariant. We see how this
can be computed to distinguish versions of the $5_2$ knot with equal classical
invariants.
\end{abstract}

\section{Introduction}
A {\it contact structure} on $\R^3$ is a method of placing a plane at every point,
such that these planes satisfy certain technical conditions.
There are various distinct contact structures we could use, some of them giving
very different properties.
However, for this paper, we will exclusively consider the standard contact structure:
At every point in $\R^3$, we consider the plane spanned by
\[\left\{\frac{\partial}{\partial y},\frac{\partial}{\partial x} + y\frac{\partial}{\partial z}\right\}.\]
Visually, we can see that these planes are always tangent to the $y$-axis, but also
twist around once when moving from $y = -\infty$ to $y = \infty$.

Now that we have this contact structure, we can define a {\it Legendrian knot}
as a knot that is at every point tangent to the plane at that point.
In particular, if we look at the tangent vector $v$ to $K$ at a point $p = (x,y,z)$,
then
\[ v = A\frac{\partial}{\partial y} + B\left(\frac{\partial}{\partial x} +y\frac{\partial}{\partial z}\right)\]
for some real numbers $A$ and $B$.
We can equivalently formulate this condition using a parametrization of $K$ as
the image of $\phi(t) = (x(t),y(t),z(t))$:
\[ z'(t)-y(t)x'(t) = 0.\]

We can now say that two Legendrian knots are equivalent if they can be related by a continuous
family of Legendrian knots. This is very similar to the definition of equivalence
for smooth knots, only here, we restrict the allowable motions.
Therefore, Legendrian equivalence is a finer relation than smooth knot equivalence,
and, in fact, we will show that there are many Legendrian inequivalent knots
with the same smooth knot type.

\section{Projections}
Before we can discuss invariants of Legendrian knots, we must first understand
the ways to visually represent them.
Because Legendrian knots rely on our choice of contact structure, and therefore
our choice of coordinates, dealing with projections of Legendrian knots will
be more complicated than with smooth knots.

The first method of projecting Legendrian knots is the {\it front projection}.
This is the projection into the $xz$-plane. From the differential equation
$z'(t) = y(t)x'(t)$, we can glean the following two properties of front projections:
\begin{enumerate}[label=\roman*)]
\item There may be no vertical tangencies: if $x'(t)$ vanishes, so does $z'(t)$. Instead we 
allow {\it cusps}, points where the diagram forms a sharp horizontal point.
\item At each crossing, the over strand has greater slope than the under strand.
\end{enumerate}
It turns out that these conditions completely characterize front projections, and
that we can arrange for the set of cusp points to be finite.

\section{Classical Invariants}

\section{Chekanov-Eliashberg DGA}

\begin{thebibliography}{9}
\bibitem{etnyre}
    J. Etnyre,
    \textit{Legendrian and Transversal Knots},
    from: Handbook of Knot Theory,
    Elsevier B. V.,
    Amsterdam (2005).

\bibitem{sabloffRulings}
    J. Sabloff,
    \textit{Augmentations and Rulings of Legendrian Knots},
    Int. Math. Res. Not. (2005) 1157-1180.

\bibitem{sabloffAMS}
    J. Sabloff,
    \textit{What Is $\ldots$ a Legendrian Knot?},
    AMS Notices, 56 (2009), no. 10, 1282-1284.
\end{thebibliography}

\end{document}
