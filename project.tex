\documentclass{article}
\usepackage{projstyle}
\usepackage{graphicx}
\usepackage{mathtools}

\title{Knot Theory Abstract}
\author{Shreyas Casturi\\Ana Fishburn}
\date{5 November 2018}

\begin{document}

\maketitle

\begin{abstract}
We start by defining Legendrian knots and isotopies of Legendrian knots.
Then we study the classical invariants of Legendrian knots:
the topological knot type, the Thurston-Bennequin number, and the rotation number.
We then explain how, while sufficient for certain classes of knots in tight
contact structures, there are nonisotopic Legendrian knots with equal
classical invariants.

In particular, we introduce the Chekanov-Eliashberg DGA, and see that its
homology, the knot contact homology, is an invariant. We see how this
can be computed to distinguish versions of the $5_2$ knot with equal classical
invariants.
\end{abstract}

\section{Introduction}
A {\it contact structure} on $\R^3$ is a method of placing a plane at every point,
such that these planes satisfy certain technical conditions.
There are various distinct contact structures we could use, some of them giving
very different properties.
However, for this paper, we will exclusively consider the standard contact structure:
At every point in $\R^3$, we consider the plane spanned by
\[\left\{\frac{\partial}{\partial y},\frac{\partial}{\partial x} + y\frac{\partial}{\partial z}\right\}.\]
Visually, we can see that these planes are always tangent to the $y$-axis, but also
twist around once when moving from $y = -\infty$ to $y = \infty$.

Now that we have this contact structure, we can define a {\it Legendrian knot}
as a knot that is at every point tangent to the plane at that point.
In particular, if we look at the tangent vector $v$ to $K$ at a point $p = (x,y,z)$,
then
\[ v = A\frac{\partial}{\partial y} + B\left(\frac{\partial}{\partial x} +y\frac{\partial}{\partial z}\right)\]
for some real numbers $A$ and $B$.
We can equivalently formulate this condition using a parametrization of $K$ as
the image of $\phi(t) = (x(t),y(t),z(t))$:
\[ z'(t)-y(t)x'(t) = 0.\]

We now define what it means for two Legendrian knots to be equivalent:
Two Legendrian knots

\section{Classical Invariants}

\section{Chekanov-Eliashberg DGA}

\begin{thebibliography}{9}
\bibitem{etnyre}
    J. Etnyre,
    \textit{Legendrian and Transversal Knots},
    from: Handbook of Knot Theory,
    Elsevier B. V.,
    Amsterdam (2005).

\bibitem{sabloffRulings}
    J. Sabloff,
    \textit{Augmentations and Rulings of Legendrian Knots},
    Int. Math. Res. Not. (2005) 1157-1180.

\bibitem{sabloffAMS}
    J. Sabloff,
    \textit{What Is $\ldots$ a Legendrian Knot?},
    AMS Notices, 56 (2009), no. 10, 1282-1284.
\end{thebibliography}

\end{document}
