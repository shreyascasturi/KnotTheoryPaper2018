\documentclass{beamer}
%\usepackage{projstyle}
\newcommand{\R}{\mathbb{R}}

\begin{document}

\title{An Introduction to Legendrian Knots}
\author{Shreyas Casturi, Ana Fishburn}
\date{28 November 2018}

\frame{\titlepage}

\section[Intro]{What are Legendrian Knots?}

\begin{frame}
    \frametitle{Contact Structures}
    \begin{definition}
    A \alert{contact structure} on $\R^3$ is a method of assigning a plane
    to every point, such that these planes satisfy certain technical conditions.
    \end{definition}

    Throughout, we will only consider the \alert{standard contact structure}, which
    twists along the $y$-axis:
\end{frame}

\begin{frame}
\frametitle{Legendrian Knots}
    \begin{definition}
    A \alert{Legendrian knot} $K$ is a smooth knot which is, at every point, parallel to the
    plane at that point given by the contact structure.
    \end{definition}

    We can use coordinates: if $K$ is the image of $t \mapsto (x(t),y(t),z(t))$, then $K$
    is Legendrian if for all $t$,
    \[ z'(t) - y(t)x'(t) = 0 \]
\end{frame}

\begin{frame}
    \frametitle{Equivalences of Legendrian Knots}
    Two Legendrian knots are equivalent if there is a continuous family of Legendrian
    knots between them. This is similar to the definition of smooth knot equivalence.
\end{frame}

\section[Proj]{Projections of Legendrian Knots}

\begin{frame}
\frametitle{This is the last slide}
\end{frame}

\section{Classical Invariants}

\end{document}
