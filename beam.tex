\documentclass{beamer}
%\usepackage{projstyle}
\newcommand{\R}{\mathbb{R}}

\begin{document}

\title{An Introduction to Legendrian Knots}
\author{Shreyas Casturi, Ana Fishburn}
\date{28 November 2018}

\frame{\titlepage}

\section[Intro]{What are Legendrian Knots?}

\begin{frame}
    \frametitle{Contact Structures}
    \begin{definition}
    A \alert{contact structure} on $\R^3$ is a method of assigning a plane
    to every point, such that these planes satisfy certain technical conditions.
    \end{definition}

    Throughout, we will only consider the \alert{standard contact structure}, which
    twists along the $y$-axis:
\end{frame}

\begin{frame}
\frametitle{Legendrian Knots}
    \begin{definition}
    A \alert{Legendrian knot} $K$ is a smooth knot which is, at every point, parallel to the
    plane at that point given by the contact structure.
    \end{definition}

    More concretely: if $K$ is the image of $\phi : [0,1] \to \R^3$ given in coordinates by
    $t \mapsto \phi(x(t),y(t),z(t))$, then $K$
    is Legendrian if for all $t$,
    \[ z'(t) - y(t)x'(t) = 0. \]
\end{frame}

\begin{frame}
    \frametitle{Equivalences of Legendrian Knots}
    Two Legendrian knots are equivalent if there is a continuous family of Legendrian
    knots between them. This is similar to the definition of smooth knot equivalence,
    only with the allowable motions restricted.

    \begin{theorem}
    If two Legendrian knots are Legendrian equivalent, then they are equivalent as smooth knots.
    \end{theorem}
\end{frame}

\section[Proj]{Projections of Legendrian Knots}

\begin{frame}
    \frametitle{Projections of Legendrian Knots}
    If we want to have clear visual representations of Legendrian knots,
    we need diagrams that respect the contact structure.

    \begin{definition}
    The \alert{front projection}, denoted \alert{$\Pi(K)$}, of a Legendrian knot $K$ is the projection
    into the $xz$-plane.
    \end{definition}

    \begin{definition}
    The \alert{Lagrangian projection}, denoted \alert{$\pi(K)$}, of a Legendrian knot $K$ is the projection
    into the $xy$-plane.
    \end{definition}
\end{frame}

\begin{frame}
    \frametitle{Properties of the Front Projection}
    We solve the equation $z'(t) - y(t)x'(t) = 0$ for $y$:
    \[ y(t) = \lim_{s\to t} \frac{z'(s)}{x'(s)}. \]

    \begin{itemize}
    \item No vertical tangencies are allowed: if $z'(t) = 0$, then $x'(t)$ must also be zero.
    Instead, we allow \alert{cusps}, where the diagram makes a sharp horizontal point.
    \item At each crossing, the under strand has greater slope than the over strand.
    \end{itemize}
\end{frame}

\begin{frame}
    \frametitle{Converting Smooth Knots into Legendrian Knots}
    Taking any diagram of a smooth knot, we can get a valid front diagram for a Legendrian
    knot by the following procedure:
    \begin{itemize}
    \item Rotate the diagram around each crossing until the under strand has greater slope.
    \item Convert every vertical tangency into a cusp.
    \end{itemize}
    \begin{theorem}
        For any smooth knot $K$, there is a Legendrian knot $K'$, such that $K$ and $K'$ are
        equivalent as smooth knots.
    \end{theorem}
\end{frame}

\begin{frame}
    \frametitle{Reidemeister Theorem for Front Projections}
    \begin{theorem}
        Two front projections represent equivalent Legendrian knots if and only if
        they can be related by a sequence of the following moves:
    \end{theorem}
\end{frame}

\begin{frame}
\frametitle{Properties of the Lagrangian Projection}
    This time, we solve the equation $z'(t) - y(t)x'(t) = 0$ for $z$:
    \[ z(t) = z_0 + \int_0^t y(s)x'(s)\,ds, \]
    so we can always recover the $z$ coordinate (up to translation) from $\pi(K)$.

    \begin{itemize}
    \item $\int_0^1 y(s)x'(s)\,ds = 0$.
    \item $\int_{t_1}^{t_2}y(s)x'(s)\,ds \neq 0$ whenever $(x(t_1),y(t_1))=(x(t_2),y(t_2))$.
    \end{itemize}
\end{frame}

\section{Classical Invariants}

\end{document}
