\documentclass{beamer}
%\usepackage{projstyle}
\newcommand{\R}{\mathbb{R}}

\begin{document}

\title{An Introduction to Legendrian Knots}
\author{Shreyas Casturi, Ana Fishburn}
\date{28 November 2018}

\frame{\titlepage}

\section[Intro]{What are Legendrian Knots?}

\begin{frame}
    \frametitle{Contact Structures}
    \begin{definition}
    A \alert{contact structure} on $\R^3$ is a method of assigning a plane
    to every point, such that these planes satisfy certain technical conditions.
    \end{definition}

    Throughout, we will only consider the \alert{standard contact structure}, which
    twists along the $y$-axis:
\end{frame}

\begin{frame}
\frametitle{Legendrian Knots}
    \begin{definition}
    A \alert{Legendrian knot} $K$ is a smooth knot which is, at every point, parallel to the
    plane at that point given by the contact structure.
    \end{definition}

    We can use coordinates: if $K$ is the image of $t \mapsto (x(t),y(t),z(t))$, then $K$
    is Legendrian if for all $t$,
    \[ z'(t) - y(t)x'(t) = 0. \]
\end{frame}

\begin{frame}
    \frametitle{Equivalences of Legendrian Knots}
    Two Legendrian knots are equivalent if there is a continuous family of Legendrian
    knots between them. This is similar to the definition of smooth knot equivalence,
    only with the allowable motions restricted.

    \begin{theorem}
    If two Legendrian knots are Legendrian equivalent, then they are equivalent as smooth knots.
    \end{theorem}
\end{frame}

\section[Proj]{Projections of Legendrian Knots}

\begin{frame}
    \frametitle{Projections of Legendrian Knots}
    If we want to have clear visual representations of Legendrian knots,
    we need diagrams that respect the contact structure.

    \begin{definition}
    The \alert{front projection} of a Legendrian knot $K$ is the projection
    into the $xz$-plane.
    \end{definition}

    \begin{definition}
    The \alert{Lagrangian projection} of a Legendrian knot $K$ is the projection
    into the $xy$-plane.
    \end{definition}
\end{frame}

\begin{frame}
    \frametitle{Properties of the Front Projection}
    To determine what properties front projections must have, look at the
    formula $x'(t) - y(t)z'(t) = 0$.
    \begin{itemize}
    \item No vertical tangencies are allowed: if $z'(t) = 0$, then $x'(t)$ is also zero.
    Instead, we allow \alert{cusps}, where the diagram makes a sharp horizontal point.
    \item At each crossing, the under strand has greater slope than the over strand.
    \end{itemize}
\end{frame}

\begin{frame}
    \frametitle{Converting Smooth Knots into Legendrian Knots}
    Taking any diagram of a smooth knot, we can get a valid front diagram for a Legendrian
    knot by the following procedure:
    \begin{itemize}
    \item Rotate the diagram around each crossing until the under strand has greater slope.
    \item Convert every vertical tangency into a cusp.
    \end{itemize}
    \begin{theorem}
        For any smooth knot $K$, there is a Legendrian knot $K'$, such that $K$ and $K'$ are
        equivalent as smooth knots.
    \end{theorem}
\end{frame}

\begin{frame}
\frametitle{Properties of the Lagrangian Projection}
\end{frame}

\begin{frame}
    \frametitle{On Classical Invariants}
    We consider the idea of an \textit{knot invariant}, and wish to
    apply this idea to \textit{Legendrian knots.}

    There are three invariants we will be discussing:
    \begin{itemize}
        \item \textit{The Topological Knot Type}
        \item \textit{The Thurston-Bennequin Number}
        \item \textit{The Rotation Number}
    \end{itemize}
    \alert{NOTE}: We will consider in the following slides knots
    that have an \textit{orientation}.
\end{frame}

\begin{frame}
    \frametitle{The Topological Knot Type}
    One invariant (or collection of invariants) we can consider
    is those invariants for \textit{smooth knots}. \\
    That is, we may consider our projections of the Legendrian knots
    as projections of smooth knots, and see what invariants hold
    between these "smooth" knots. \\
    The \textit{topological knot type} is then the type of knot we find
    when we consider the knots as a \textit{smooth} knots and not as 
    \textit{Legendrian} knots. \\
   
\end{frame}

\begin{frame}
    \frametitle{The Thurston-Bennequin Invariant}
    \begin{definition}
    This invariant measures the degree of twisting (twisting of planes)
    around our given Legendrian knot, which we denote as $L.$
\end{definition}
The formal definition involves heavy usage of geometrical concepts,
so we'll revert to a combinatorial, calculation-based method of
obtaining this invariant.
\end{frame}
\begin{frame}
    \frametitle{An Example of Calculating TB (Front)}
    For a \textit{front} projection:
    \begin{itemize}
        \item There are two things we must calculate:
        \begin{itemize}
            \item The \textit{writhe} number of the knot projection.
            \item The number of \textit{cusps} the knot projection has.
        \end{itemize}
    \end{itemize}
    The formula for a \textit{front} projection:
    \[tb(L) = writhe(\Pi(L)) - \frac{1}{2}(cusps)\]
    There are two things we must calculate for a \textit{Legendrian knot}:
\end{frame}

\begin{frame}
    \frametitle{An Example of Calculating TB (Lagrangian)}
    For a \textit{Lagrangian} projection, we need only the \textit{writhe}
    number of the projection. \\
    The formula is literally:
    \[tb(L) = writhe(\pi(L))\].
\end{frame}

\begin{frame}
    \frametitle{The Rotation Number}
    \begin{definition}
        The \alert{winding number} of a curve around some given point
        is the number of times the curve travels \textit{counterclockwise}
        around the point.
    \end{definition}

    \begin{definition}
        The \alert{rotation number} is the winding number of a \textit{Legendrian
          knot}. The point to travel around is the origin of $\mathbb{R^2}.$
    \end{definition}
    As with $tb$, the reasoning of why the rotation number is the winding
    number involves higher-level geometric concepts, so we'll defer
    to a calculation to explain the rotation number.
\end{frame}


\begin{frame}
    \frametitle{Calculating Rotation Number}
    With a \textit{front projection}:
    \begin{itemize}
        \item{We need the amount of \textit{down} cusps a projection has, denoted
        as \textit{D}.}
      \item{We also need the amount of \textit{up} cusps a projection has,
      denoted as \textit{U}.}
    \end{itemize}
    With this, our formula is:
    \[r(L) = \frac{1}{2}(D - U).\]
    With a \textit{Lagrangian projection}:
    \[r(L) = winding(\pi(L)).\]
    \alert{NOTE}: The orientation can change the sign of the rotation number.
\end{frame}

\begin{frame}
    \frametitle{An Example of Calculating Rotation Number}
    Insert diagram here.
\end{frame}


     
    


\end{document}
